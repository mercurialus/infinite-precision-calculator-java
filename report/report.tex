\documentclass[12pt]{article}
\usepackage{graphicx}
\usepackage{listings}
\usepackage{hyperref}
\usepackage{geometry}
\usepackage{enumitem}
\usepackage{hyphenat}
\geometry{margin=1in}

\title{Arbitrary Precision Arithmetic Library\\Software Development Fundamentals (CS1023)\\\vspace{0.5em}Project Report}
\author{Harshil Goyal}
\date{\today}

% Code listing style
\lstset{
  basicstyle=\ttfamily\footnotesize,
  breaklines=true,
  columns=fullflexible,
  frame=single,
  captionpos=b
}

\begin{document}
\maketitle
\tableofcontents
\newpage

%============================================================
\section{Introduction}
This report outlines the design and development of a Java library that supports \textbf{arbitrary-precision arithmetic} for both integers and floating-point numbers. The project addresses the first project from the CS1023 “Software Development Fundamentals” course (Jan–May~2025). The final submission includes:

\begin{enumerate}[label=\arabic*.]
  \item A compiled JAR file located at \texttt{src/arbitraryarithmetic/aarithmetic.jar}, containing the public classes \texttt{AInteger} and \texttt{AFloat}.
  \item A command-line utility named \texttt{MyInfArith} that can evaluate a single binary arithmetic expression.
  \item An Ant build script (\texttt{build.xml}) for automating compilation, packaging, and execution.
  \item This report, written in \LaTeX.
\end{enumerate}

%============================================================
\section{Design}
\subsection{High-level Overview}
The library internally represents numbers in \emph{base 10000}. Each array element stores four decimal digits, offering a trade-off between memory efficiency and ease of implementing basic arithmetic algorithms. The UML diagram in Figure~\ref{fig:uml} illustrates the class structure.

\begin{figure}[h]
  \centering
  \includegraphics[width=0.9\linewidth]{uml_sdf1.png}
  \caption{UML class diagram of the package \texttt{arbitraryarithmetic}.}
  \label{fig:uml}
\end{figure}

\paragraph{Core Design Principles}
\begin{itemize}
  \item \textbf{Immutable API}: All public methods return new instances. Internal mutations are hidden to maintain referential transparency.
  \item \textbf{Sign handling}: A dedicated boolean field, \texttt{isNegative}, avoids the complexity of two’s complement arithmetic.
  \item \textbf{Floating-point structure}: Each \texttt{AFloat} object consists of a pair \((\textit{unscaled},\,\textit{scale})\), following the model $\textit{value}=\textit{unscaled}\times10^{-\textit{scale}}$.
  \item \textbf{Precision guarantee}: All exact digits are preserved in every calculation. The \texttt{toString()} method simply truncates the fractional portion to 30 digits for display purposes.
\end{itemize}

\subsection{Algorithmic Choices}
\begin{description}[style=nextline]
  \item[Addition / Subtraction] Implemented via straightforward digit-by-digit traversal with carry or borrow logic. Sign processing is handled separately.
  \item[Multiplication] A classic \(O(n^2)\) approach is used, it's a classic middle-school multiplication approach.
  \item[Division] Performed using long division, with each digit found through binary search in the range 0–9999. This avoids floating-point math.
  \item[Floating-point Operations] Operands are rescaled to a shared exponent before addition or subtraction. Multiplication adds exponents; division increases the dividend's scale to ensure precision.
\end{description}

%============================================================
\section{Implementation Details}
\subsection{\texttt{AInteger}}
\begin{itemize}
  \item Stores digits in \texttt{java.util.ArrayList<Integer> value}, with the least significant digit first.
  \item Includes constructors, parsing logic, and basic arithmetic operations.
  \item Internal helpers like \texttt{compareAbsolute}, \texttt{addAbsolute}, and \texttt{subAbsolute} are package-private to support reuse in \texttt{AFloat}.
\end{itemize}

\subsection{\texttt{AFloat}}
\begin{itemize}
  \item Combines an \texttt{AInteger unscaled} value with an integer \texttt{scale}.
  \item Normalizes results via the \texttt{stripZeros()} method to maintain a consistent format.
  \item Ensures that the decimal output is accurate up to 30 digits.
\end{itemize}

\subsection{\texttt{MyInfArith}}
This is a minimal command-line interface that maps user input to the appropriate method in the library. It serves both as a usage demonstration and a basic validation tool.

\subsection{Build Automation (Ant)}
The Ant script \texttt{build.xml} includes targets for \texttt{clean}, \texttt{compile}, \texttt{jar}, and \texttt{run}. To execute an operation, users can run:

\lstset{showstringspaces=false}
\begin{lstlisting}[language=bash]
ant run -Dargs="int add 2 3"
\end{lstlisting}

from the project’s root directory.

%============================================================
\section{User Guide (README)}
\subsection{Compiling}\label{sec:compile}
\begin{enumerate}
  \item Install JDK 17 (or newer) and Apache Ant.
  \item Clone the project repository and navigate to its root directory.
  \item Run \lstinline|ant jar| to generate \texttt{arbitraryarithmetic/aarithmetic.jar} inside the \texttt{src/} folder.
\end{enumerate}

\subsection{Running the CLI}
Command format:
\begin{lstlisting}[language=bash]
python run_project.py build | clean | <int|float> <add|sub|mul|div> <operand1> <operand2>
\end{lstlisting}

Example:
\begin{lstlisting}[language=bash]
$ python run_project.py float div 244727.15202 75964.3891
3.221603634537752111008551505615
\end{lstlisting}

%============================================================
\subsection{Using Docker}
If you prefer to use docker, you pull the docker image and run the project on your system.
Command format:
\begin{lstlisting}[language=bash]
docker image pull mercurialus/inf-cal
docker run -it mercurialus/inf-cal
\end{lstlisting}

Example:
\begin{lstlisting}[language=bash]
root@921168cdd6d4:/app# python run_project.py float div 244727.15202 75964.3891
3.221603634537752111008551505615
\end{lstlisting}



%============================================================
\section{Limitations}
\begin{itemize}
  \item The current multiplication algorithm has a time complexity of \(O(n^2)\); division operates at \(O(n^2 \)). These are acceptable for coursework but inefficient for large inputs.
  \item Only truncation is used for rounding; other modes like rounding up or to nearest even are not supported.
\end{itemize}

%============================================================
\section{Git Commit Snapshot}
\begin{figure}[h]
  \centering
  \includegraphics[width=0.9\linewidth]{image.png}
  \caption{Snapshot of the Git commit history.}
  \label{fig:git}
\end{figure}

%============================================================
\section{Key Learnings}
\begin{itemize}
  \item Selecting an internal numeric base impacts both computational efficiency and string formatting.
  \item Managing sign propagation correctly is essential for clean and bug-free arithmetic.
  \item Writing tests alongside development speeds up debugging and ensures correctness.
  \item Dockerization of the project simplifies deployment and testing across different environments.
  \item Desigining a command-line interface requires careful consideration of user input and error handling.
  \item Using Ant for build automation streamlines the development process and reduces manual errors.
  \item Understanding the limitations of algorithms is crucial for making informed design choices.
  \item Automating builds using Ant simplifies project maintenance and reproducibility.
\end{itemize}

%============================================================
\section{Conclusion}
This project delivers a fully functional Java library for arbitrary-precision arithmetic, adhering to all specifications and reflecting sound software engineering principles.

\end{document}
